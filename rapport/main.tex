\documentclass[11pt,letterpaper]{article}

% ============================================
% PACKAGES
% ============================================
\usepackage[utf8]{inputenc}
\usepackage[T1]{fontenc}
\usepackage{geometry}
\usepackage{amsmath,amssymb,amsthm}
\usepackage{enumitem}
\usepackage{fancyhdr}
\usepackage{lastpage}
\usepackage{graphicx}
\usepackage{float}
\usepackage{xcolor}
\usepackage{tcolorbox}
\usepackage{hyperref}

% ============================================
% PAGE SETUP
% ============================================
\geometry{
    top=2cm,
    bottom=2cm,
    left=2.5cm,
    right=2.5cm
}

% ============================================
% HEADER AND FOOTER
% ============================================
\pagestyle{fancy}
\fancyhf{}
\lhead{\textbf{\coursename}}
\rhead{\textbf{\studentname}}
\cfoot{Page \thepage\ of \pageref{LastPage}}
\renewcommand{\headrulewidth}{0.4pt}
\renewcommand{\footrulewidth}{0.4pt}

% ============================================
% HOMEWORK INFO (EDIT THESE)
% ============================================
\newcommand{\coursename}{MOGPL}
\newcommand{\studentname}{Bruno Fernandes Iorio \& Gildas}

% ============================================
% CUSTOM ENVIRONMENTS
% ============================================

% Problem environment
\newcounter{problemcounter}
\newenvironment{problem}[1][]{%
    \refstepcounter{problemcounter}%
    \par\vspace{1em}%
    \noindent\textbf{Problem \theproblemcounter\ifx&#1&\else: #1\fi}%
    \par\vspace{0.5em}%
}{\par\vspace{1em}}

% Solution environment
\newenvironment{solution}{%
    \par\vspace{0.5em}%
    \noindent\textit{Solution:}%
    \par\vspace{0.3em}%
}{\par\vspace{1em}}

% Boxed solution for final answers
\newtcolorbox{answerbox}{
    colback=blue!5!white,
    colframe=blue!50!black,
    boxrule=0.5pt,
    arc=2pt,
    left=5pt,
    right=5pt,
    top=5pt,
    bottom=5pt
}

% ============================================
% MATH SHORTCUTS
% ============================================
\newcommand{\R}{\mathbb{R}}
\newcommand{\N}{\mathbb{N}}
\newcommand{\Z}{\mathbb{Z}}
\newcommand{\Q}{\mathbb{Q}}
\newcommand{\C}{\mathbb{C}}
\newcommand{\dd}{\,\mathrm{d}}
\newcommand{\pd}[2]{\frac{\partial #1}{\partial #2}}
\newcommand{\der}[2]{\frac{\mathrm{d} #1}{\mathrm{d} #2}}

% ============================================
% DOCUMENT
% ============================================
\begin{document}

% ============================================
% TITLE SECTION
% ============================================
\begin{center}
    {\Large\textbf{\coursename}}\\[0.3em]
    \begin{tabular}{rl}
        \textbf{Name:} & \studentname \\
    \end{tabular}
\end{center}

\hrule
\vspace{1em}

% ============================================
% PROBLEMS
% ============================================
\noindent
\textbf{(a)} 
Soit $G$ l'ensemble de sommets du grille $N$ x $M$ et $O \subset G$ l'ensemble de sommets obstacles. Pour chaque point $(x,y) \in G$ et chaque direction $dir$, on considere le sommet $(x,y,dir)$, en indicant que 
qu'on face vers $dir$ au point $(x,y)$. La commande avance($k$) est representee par l'arc orientee entre deux sommet avec la meme direction et une distance $k\in\{1,2,3\}$. 
La commande tourne est represantee par l'arc 
orientee entre deux sommets avec les memes coordonnees $x, y$, mais des directions differentes.

Pour avoir que chaque commande aie un cout unitaire, on definit les arc sortantes suivantes pour chaque $(x,y) \notin O$:
\begin{itemize}
  \item $(x,y,sud) \longrightarrow  \{(x,y,ouest), (x,y,est)\} \cup \{(x,y+i,sud) : i \in \{1,2,3\}, M > y+i \geq 0, (x, y+i) \notin O\}$
  \item $(x,y,nord) \longrightarrow  \{(x,y,est), (x,y,ouest)\} \cup \{(x,y-i,nord) : i \in \{1,2,3\}, M > y-i \geq 0, (x, y-i) \notin O\}$
  \item $(x,y,est) \longrightarrow  \{(x,y,sud), (x,y,nord)\} \cup \{(x+i,y,est) : i \in \{1,2,3\}, N > x+i \geq 0, (x+i, y) \notin O\}$
  \item $(x,y,ouest) \longrightarrow  \{(x,y,nord), (x,y,sud)\} \cup \{(x-i,y,ouest) : i \in \{1,2,3\}, N > x-i \geq 0, (x-i, y) \notin O\}$
\end{itemize}

\end{document}
